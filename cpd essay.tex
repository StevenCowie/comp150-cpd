% Please do not change the document class
\documentclass{scrartcl}

% Please do not change these packages
\usepackage[hidelinks]{hyperref}
\usepackage[none]{hyphenat}
\usepackage{setspace}
\doublespace

% You may add additional packages here
\usepackage{amsmath}

% Please include a clear, concise, and descriptive title
\title{CPD Report}

% Please do not change the subtitle
\subtitle{COMP150 - CPD Report}

% Please put your student number in the author field
\author{1605240}

\begin{document}

\maketitle

\section*{Introduction}
Ever since I was a young child I've always dreamed of being able to make video games for a living. I'd love to join up or form a small indie game development team then eventually work on triple AAA. During my first term I've identified areas and skills that I need to improve on for me to reach my goals. They are my essay writing skills, my use of version control, my presenting skills, prioritizing tasks and lastly independent studying.

\section{Essay Writing}
During the term when it came to writing essays I got very nervous as I hadn't properly written one in years. Since they were supposed to be of a higher standard than secondary school I struggled as the jump was a big one. Throughout my time at university I plan on trying to do some of my own research in both the academic and games industry related field, to do this my writing style needs to change dramatically.This will also help me as I can center my research on topics we are doing to educate myself more.  It is essential to be able to convey my thoughts, ideas and findings in a way that is of a high standard as in the industry no one will want to read a paper if it isn't written of a high enough standard. To improve on this I plan to spend time during the year setting myself small projects to research into to get a better feel and more experience of the writing style I want to achieve. I plan on doing at least two term on top of the ones we are given already which I think is a realistic goal to achieve.


\section{Presenting Skills}

Throughout the term I came to realize that when it comes to presenting in front of a large group or audience I tend to freeze and forget what I'm saying and most of the time I think my ideas will not be like by others which I think plays a large part in why I don't like it. I'll happily talk to a group of about six people with no problem. On our course this can become a big problem as a large part of it is communicating ideas and presenting. This would effect my career heavily in the games industry as a large part of that is presenting your game ideas to publishers and investors and if you're not confident in your idea / game then no one else will be. To try and solve this problem I need to be more relaxed when presenting and realize that people won't judge me for my ideas, the only way to do this is by actually getting up and presenting to them. To make myself feel more comfortable I will write notes and practice it several times beforehand so I can try and memorize exactly what I'm saying so I don't freeze. I'm hoping that over the coming months I can have tackled this issue. I will volunteer to present my groups talks to help with this too.

\section{Version Control}

When working on projects I noticed I that when I was doing the work I tended to do a large chunk of it at once and leave it for a week or more in between thus making my use or version control awful. This isn't a good habit as we get marked for our use of version control on projects and tends to leave me stressed as I feel I need to get the code to work on the day or I feel as I've done something wrong. This would be frowned upon in the games industry as developers tend to upload code daily so other team members can work on it, fix it and add to it and if I only did this once a week it wouldn't be fair on my colleagues and I would be letting the team down. To fix this issue I need to set aside a certain time every day or at most every other day in which I sit down, even if it's just for an hour, in which I will work on my current projects and upload all the code I make, even if it's not finished or wrong, GitHub will help measure my progress.

\section{Prioritising Taks}

Prioritising tasks was another issue I found during the term. I tended to do the easiest tasks first, even is this meant leaving the tasks with deadlines coming up sooner if the work was harder. This caused some of my work being a higher standard as I had longer to do the easier tasks and a short amount of time to do the tasks I found harder. I think this all links back to my fear of getting things wrong since I'd rather not answer than get something wrong, this is a big flaw of mine that urgently needs to change. In the game development industry I could not do this as it is highly important to prioritise your tasks properly and choose the ones that need to get done first. If I didn't do this it could lead to my team members having to compensate and do some of the work for me. To help me with this problem I plan on getting the visual aid of a wall calendar to help me plan what I need to do and on what night it needs to be done. This will help me because I tend to actually do tasks when I can physically see they need to be done. To measure this I plan on doing the work I get first and sticking to that.

\section{Independent Studying}

When it came to independent studying I was very inconsistent, some weeks I would do a few hours and other I would struggle to do one. This has lead to inconsistencies in my ability to code because for the topics I am enjoy I know what I'm doing quite well and for the ones I struggled with I don't  know what I'm doing as well as I should. This would not help me get a job in the games industry as employers like to look for programmers that balanced knowledge of more topics rather than someone who knows a lot more about others and not so much about them all. To fix this issue I plan on doing at least an hour of programming every night and if I miss out on one night to do two the next. I will focus this on the topics we are studying at the time, topics I am struggling with and topics I have an interest in learning. This ties into my point before with the calendar, I will be able to write down for the week what I plan to study. To measure this I plan on setting myself small projects to do, if the code works I will have succeeded.

\section*{Conclusion}

I'm hoping that by the end of the academic year I myself will have noticed a large improvement in my ability and my work produced. To do this I need to specifically focus on the five key skills listed above and learn to master them all.  By doing so it I'm hoping it will make me a better student and increase my employability prospects for when I leave university. 

\bibliographystyle{ieeetran}
\bibliography{references}

\end{document}
